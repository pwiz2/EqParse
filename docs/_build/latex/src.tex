% Generated by Sphinx.
\def\sphinxdocclass{report}
\documentclass[letterpaper,10pt,english]{sphinxmanual}
\usepackage[utf8]{inputenc}
\DeclareUnicodeCharacter{00A0}{\nobreakspace}
\usepackage{cmap}
\usepackage[T1]{fontenc}
\usepackage{babel}
\usepackage{times}
\usepackage[Bjarne]{fncychap}
\usepackage{longtable}
\usepackage{sphinx}
\usepackage{multirow}


\title{EqParse Documentation}
\date{December 16, 2014}
\release{}
\author{Haroon Arshad}
\newcommand{\sphinxlogo}{}
\renewcommand{\releasename}{Release}
\makeindex

\makeatletter
\def\PYG@reset{\let\PYG@it=\relax \let\PYG@bf=\relax%
    \let\PYG@ul=\relax \let\PYG@tc=\relax%
    \let\PYG@bc=\relax \let\PYG@ff=\relax}
\def\PYG@tok#1{\csname PYG@tok@#1\endcsname}
\def\PYG@toks#1+{\ifx\relax#1\empty\else%
    \PYG@tok{#1}\expandafter\PYG@toks\fi}
\def\PYG@do#1{\PYG@bc{\PYG@tc{\PYG@ul{%
    \PYG@it{\PYG@bf{\PYG@ff{#1}}}}}}}
\def\PYG#1#2{\PYG@reset\PYG@toks#1+\relax+\PYG@do{#2}}

\def\PYG@tok@gd{\def\PYG@tc##1{\textcolor[rgb]{0.63,0.00,0.00}{##1}}}
\def\PYG@tok@gu{\let\PYG@bf=\textbf\def\PYG@tc##1{\textcolor[rgb]{0.50,0.00,0.50}{##1}}}
\def\PYG@tok@gt{\def\PYG@tc##1{\textcolor[rgb]{0.00,0.25,0.82}{##1}}}
\def\PYG@tok@gs{\let\PYG@bf=\textbf}
\def\PYG@tok@gr{\def\PYG@tc##1{\textcolor[rgb]{1.00,0.00,0.00}{##1}}}
\def\PYG@tok@cm{\let\PYG@it=\textit\def\PYG@tc##1{\textcolor[rgb]{0.25,0.50,0.56}{##1}}}
\def\PYG@tok@vg{\def\PYG@tc##1{\textcolor[rgb]{0.73,0.38,0.84}{##1}}}
\def\PYG@tok@m{\def\PYG@tc##1{\textcolor[rgb]{0.13,0.50,0.31}{##1}}}
\def\PYG@tok@mh{\def\PYG@tc##1{\textcolor[rgb]{0.13,0.50,0.31}{##1}}}
\def\PYG@tok@cs{\def\PYG@tc##1{\textcolor[rgb]{0.25,0.50,0.56}{##1}}\def\PYG@bc##1{\colorbox[rgb]{1.00,0.94,0.94}{##1}}}
\def\PYG@tok@ge{\let\PYG@it=\textit}
\def\PYG@tok@vc{\def\PYG@tc##1{\textcolor[rgb]{0.73,0.38,0.84}{##1}}}
\def\PYG@tok@il{\def\PYG@tc##1{\textcolor[rgb]{0.13,0.50,0.31}{##1}}}
\def\PYG@tok@go{\def\PYG@tc##1{\textcolor[rgb]{0.19,0.19,0.19}{##1}}}
\def\PYG@tok@cp{\def\PYG@tc##1{\textcolor[rgb]{0.00,0.44,0.13}{##1}}}
\def\PYG@tok@gi{\def\PYG@tc##1{\textcolor[rgb]{0.00,0.63,0.00}{##1}}}
\def\PYG@tok@gh{\let\PYG@bf=\textbf\def\PYG@tc##1{\textcolor[rgb]{0.00,0.00,0.50}{##1}}}
\def\PYG@tok@ni{\let\PYG@bf=\textbf\def\PYG@tc##1{\textcolor[rgb]{0.84,0.33,0.22}{##1}}}
\def\PYG@tok@nl{\let\PYG@bf=\textbf\def\PYG@tc##1{\textcolor[rgb]{0.00,0.13,0.44}{##1}}}
\def\PYG@tok@nn{\let\PYG@bf=\textbf\def\PYG@tc##1{\textcolor[rgb]{0.05,0.52,0.71}{##1}}}
\def\PYG@tok@no{\def\PYG@tc##1{\textcolor[rgb]{0.38,0.68,0.84}{##1}}}
\def\PYG@tok@na{\def\PYG@tc##1{\textcolor[rgb]{0.25,0.44,0.63}{##1}}}
\def\PYG@tok@nb{\def\PYG@tc##1{\textcolor[rgb]{0.00,0.44,0.13}{##1}}}
\def\PYG@tok@nc{\let\PYG@bf=\textbf\def\PYG@tc##1{\textcolor[rgb]{0.05,0.52,0.71}{##1}}}
\def\PYG@tok@nd{\let\PYG@bf=\textbf\def\PYG@tc##1{\textcolor[rgb]{0.33,0.33,0.33}{##1}}}
\def\PYG@tok@ne{\def\PYG@tc##1{\textcolor[rgb]{0.00,0.44,0.13}{##1}}}
\def\PYG@tok@nf{\def\PYG@tc##1{\textcolor[rgb]{0.02,0.16,0.49}{##1}}}
\def\PYG@tok@si{\let\PYG@it=\textit\def\PYG@tc##1{\textcolor[rgb]{0.44,0.63,0.82}{##1}}}
\def\PYG@tok@s2{\def\PYG@tc##1{\textcolor[rgb]{0.25,0.44,0.63}{##1}}}
\def\PYG@tok@vi{\def\PYG@tc##1{\textcolor[rgb]{0.73,0.38,0.84}{##1}}}
\def\PYG@tok@nt{\let\PYG@bf=\textbf\def\PYG@tc##1{\textcolor[rgb]{0.02,0.16,0.45}{##1}}}
\def\PYG@tok@nv{\def\PYG@tc##1{\textcolor[rgb]{0.73,0.38,0.84}{##1}}}
\def\PYG@tok@s1{\def\PYG@tc##1{\textcolor[rgb]{0.25,0.44,0.63}{##1}}}
\def\PYG@tok@gp{\let\PYG@bf=\textbf\def\PYG@tc##1{\textcolor[rgb]{0.78,0.36,0.04}{##1}}}
\def\PYG@tok@sh{\def\PYG@tc##1{\textcolor[rgb]{0.25,0.44,0.63}{##1}}}
\def\PYG@tok@ow{\let\PYG@bf=\textbf\def\PYG@tc##1{\textcolor[rgb]{0.00,0.44,0.13}{##1}}}
\def\PYG@tok@sx{\def\PYG@tc##1{\textcolor[rgb]{0.78,0.36,0.04}{##1}}}
\def\PYG@tok@bp{\def\PYG@tc##1{\textcolor[rgb]{0.00,0.44,0.13}{##1}}}
\def\PYG@tok@c1{\let\PYG@it=\textit\def\PYG@tc##1{\textcolor[rgb]{0.25,0.50,0.56}{##1}}}
\def\PYG@tok@kc{\let\PYG@bf=\textbf\def\PYG@tc##1{\textcolor[rgb]{0.00,0.44,0.13}{##1}}}
\def\PYG@tok@c{\let\PYG@it=\textit\def\PYG@tc##1{\textcolor[rgb]{0.25,0.50,0.56}{##1}}}
\def\PYG@tok@mf{\def\PYG@tc##1{\textcolor[rgb]{0.13,0.50,0.31}{##1}}}
\def\PYG@tok@err{\def\PYG@bc##1{\fcolorbox[rgb]{1.00,0.00,0.00}{1,1,1}{##1}}}
\def\PYG@tok@kd{\let\PYG@bf=\textbf\def\PYG@tc##1{\textcolor[rgb]{0.00,0.44,0.13}{##1}}}
\def\PYG@tok@ss{\def\PYG@tc##1{\textcolor[rgb]{0.32,0.47,0.09}{##1}}}
\def\PYG@tok@sr{\def\PYG@tc##1{\textcolor[rgb]{0.14,0.33,0.53}{##1}}}
\def\PYG@tok@mo{\def\PYG@tc##1{\textcolor[rgb]{0.13,0.50,0.31}{##1}}}
\def\PYG@tok@mi{\def\PYG@tc##1{\textcolor[rgb]{0.13,0.50,0.31}{##1}}}
\def\PYG@tok@kn{\let\PYG@bf=\textbf\def\PYG@tc##1{\textcolor[rgb]{0.00,0.44,0.13}{##1}}}
\def\PYG@tok@o{\def\PYG@tc##1{\textcolor[rgb]{0.40,0.40,0.40}{##1}}}
\def\PYG@tok@kr{\let\PYG@bf=\textbf\def\PYG@tc##1{\textcolor[rgb]{0.00,0.44,0.13}{##1}}}
\def\PYG@tok@s{\def\PYG@tc##1{\textcolor[rgb]{0.25,0.44,0.63}{##1}}}
\def\PYG@tok@kp{\def\PYG@tc##1{\textcolor[rgb]{0.00,0.44,0.13}{##1}}}
\def\PYG@tok@w{\def\PYG@tc##1{\textcolor[rgb]{0.73,0.73,0.73}{##1}}}
\def\PYG@tok@kt{\def\PYG@tc##1{\textcolor[rgb]{0.56,0.13,0.00}{##1}}}
\def\PYG@tok@sc{\def\PYG@tc##1{\textcolor[rgb]{0.25,0.44,0.63}{##1}}}
\def\PYG@tok@sb{\def\PYG@tc##1{\textcolor[rgb]{0.25,0.44,0.63}{##1}}}
\def\PYG@tok@k{\let\PYG@bf=\textbf\def\PYG@tc##1{\textcolor[rgb]{0.00,0.44,0.13}{##1}}}
\def\PYG@tok@se{\let\PYG@bf=\textbf\def\PYG@tc##1{\textcolor[rgb]{0.25,0.44,0.63}{##1}}}
\def\PYG@tok@sd{\let\PYG@it=\textit\def\PYG@tc##1{\textcolor[rgb]{0.25,0.44,0.63}{##1}}}

\def\PYGZbs{\char`\\}
\def\PYGZus{\char`\_}
\def\PYGZob{\char`\{}
\def\PYGZcb{\char`\}}
\def\PYGZca{\char`\^}
\def\PYGZsh{\char`\#}
\def\PYGZpc{\char`\%}
\def\PYGZdl{\char`\$}
\def\PYGZti{\char`\~}
% for compatibility with earlier versions
\def\PYGZat{@}
\def\PYGZlb{[}
\def\PYGZrb{]}
\makeatother

\begin{document}

\maketitle
\tableofcontents
\phantomsection\label{index::doc}


Contents:


\chapter{eqparse package}
\label{eqparse::doc}\label{eqparse:welcome-to-src-s-documentation}\label{eqparse:eqparse-package}

\section{\texttt{eqparse} Package}
\label{eqparse:id1}
The equation parse package is a universal parser from a single-input file describing a larger ODE system to different programming languages. Multiple file output of the same language can easily be created and other parser modules can easily be implemented for other language outputs with some python knowledge
\phantomsection\label{eqparse:module-eqparse.__init__}\index{eqparse.\_\_init\_\_ (module)}
eqparse import module initialise


\section{\texttt{baseparse} Class}
\label{eqparse:baseparse-class}\index{baseparse (in module eqparse.\_\_init\_\_)}

\begin{fulllineitems}
\phantomsection\label{eqparse:eqparse.__init__.baseparse}\pysigline{\code{eqparse.\_\_init\_\_.}\bfcode{baseparse}}
alias of {\hyperref[eqparse:module-eqparse.baseparse]{\code{eqparse.baseparse}}}

\end{fulllineitems}



\section{\texttt{baseparse} module}
\label{eqparse:module-eqparse.baseparse}\label{eqparse:baseparse-module}\index{eqparse.baseparse (module)}\index{BaseParse (class in eqparse.baseparse)}

\begin{fulllineitems}
\phantomsection\label{eqparse:eqparse.baseparse.BaseParse}\pysigline{\strong{class }\code{eqparse.baseparse.}\bfcode{BaseParse}}
Bases: \code{object}

The main objective of {\hyperref[eqparse:eqparse.baseparse.BaseParse]{\code{BaseParse}}} is to organise universal
functions and member variables that are inherited in other class 
parser modules. Current supplied parsers include the \code{MatlabParser},
\code{CppParser} and \code{XPPautParser} as child modules.

Initialise universal variables usually important in most or all
language syntax of the following. if a module inherits this class, 
the following variables usually need to be redefined according to its
language syntax. The following parameters are automatically defined as
\_\_init\_\_ is called
\begin{quote}\begin{description}
\item[{Parameters}] \leavevmode\begin{itemize}
\item {} 
\textbf{l\_enclose} -- The left character to access a piece of memory is usually a left square bracket (C++) or a  normal left bracket (MATLAB

\item {} 
\textbf{r\_enclose} -- Similar to :param l\_enclose:

\item {} 
\textbf{comment\_prefix} -- character used for single-line comment

\item {} 
\textbf{vec\_counter\_start} -- starting number to access first element of a container/array (default: 0)

\item {} 
\textbf{container\_type} -- a list of container types used

\item {} 
\textbf{PERM\_I\_LIST} -- the input file contain preconditioned statements in csv input file such as if statements or power function.

\item {} 
\textbf{operations} -- the required result during parse of the :param PERM\_I\_LIST: list to the respective language syntax

\end{itemize}

\end{description}\end{quote}
\index{close\_file() (eqparse.baseparse.BaseParse method)}

\begin{fulllineitems}
\phantomsection\label{eqparse:eqparse.baseparse.BaseParse.close_file}\pysiglinewithargsret{\bfcode{close\_file}}{\emph{file\_n\_str}}{}
\end{fulllineitems}

\index{end() (eqparse.baseparse.BaseParse method)}

\begin{fulllineitems}
\phantomsection\label{eqparse:eqparse.baseparse.BaseParse.end}\pysiglinewithargsret{\bfcode{end}}{}{}
\end{fulllineitems}

\index{get\_directory() (eqparse.baseparse.BaseParse method)}

\begin{fulllineitems}
\phantomsection\label{eqparse:eqparse.baseparse.BaseParse.get_directory}\pysiglinewithargsret{\bfcode{get\_directory}}{}{}
\end{fulllineitems}

\index{get\_index() (eqparse.baseparse.BaseParse method)}

\begin{fulllineitems}
\phantomsection\label{eqparse:eqparse.baseparse.BaseParse.get_index}\pysiglinewithargsret{\bfcode{get\_index}}{\emph{index\_to\_get}}{}
\end{fulllineitems}

\index{get\_list() (eqparse.baseparse.BaseParse method)}

\begin{fulllineitems}
\phantomsection\label{eqparse:eqparse.baseparse.BaseParse.get_list}\pysiglinewithargsret{\bfcode{get\_list}}{\emph{mKey}, \emph{mLib=None}, \emph{mPrefix='`}, \emph{mPostfix='`}, \emph{mIndex=None}}{}
get a list of a particular property of the variables. The list of properties depends on a key which associates a particular property of a variables library (e.g. comment, rhs).
\begin{quote}\begin{description}
\item[{Parameters}] \leavevmode\begin{itemize}
\item {} 
\textbf{mKey} -- the property type of a variable

\item {} 
\textbf{mLib} -- library to use e.g self.data (normal format) self.data\_vec (vectorised format) (default - self.data)

\item {} 
\textbf{mPrefix} -- pre-fix the returned property with a string.

\item {} 
\textbf{mPostfix} -- post-fix the returned property with a string.

\end{itemize}

\end{description}\end{quote}

\end{fulllineitems}

\index{get\_names() (eqparse.baseparse.BaseParse method)}

\begin{fulllineitems}
\phantomsection\label{eqparse:eqparse.baseparse.BaseParse.get_names}\pysiglinewithargsret{\bfcode{get\_names}}{}{}
\end{fulllineitems}

\index{get\_only\_names() (eqparse.baseparse.BaseParse method)}

\begin{fulllineitems}
\phantomsection\label{eqparse:eqparse.baseparse.BaseParse.get_only_names}\pysiglinewithargsret{\bfcode{get\_only\_names}}{\emph{oNames}, \emph{mIndex}}{}
\end{fulllineitems}

\index{initialise\_library() (eqparse.baseparse.BaseParse method)}

\begin{fulllineitems}
\phantomsection\label{eqparse:eqparse.baseparse.BaseParse.initialise_library}\pysiglinewithargsret{\bfcode{initialise\_library}}{\emph{lib}}{}
In order to initialise library
\begin{quote}\begin{description}
\item[{Parameters}] \leavevmode
\textbf{lib} -- The created library from the \code{createlibrary} class

\item[{Returns}] \leavevmode
void

\end{description}\end{quote}

\end{fulllineitems}

\index{new\_dependancy\_index() (eqparse.baseparse.BaseParse method)}

\begin{fulllineitems}
\phantomsection\label{eqparse:eqparse.baseparse.BaseParse.new_dependancy_index}\pysiglinewithargsret{\bfcode{new\_dependancy\_index}}{\emph{index\_to\_order}, \emph{key\_to\_order}, \emph{is\_return=True}}{}
\end{fulllineitems}

\index{new\_index() (eqparse.baseparse.BaseParse method)}

\begin{fulllineitems}
\phantomsection\label{eqparse:eqparse.baseparse.BaseParse.new_index}\pysiglinewithargsret{\bfcode{new\_index}}{\emph{*argv}}{}
\end{fulllineitems}

\index{new\_names() (eqparse.baseparse.BaseParse method)}

\begin{fulllineitems}
\phantomsection\label{eqparse:eqparse.baseparse.BaseParse.new_names}\pysiglinewithargsret{\bfcode{new\_names}}{\emph{oNames, mIndex, specifier={[}'`, `'{]}, return\_it=True}}{}
\end{fulllineitems}

\index{new\_names\_restricted() (eqparse.baseparse.BaseParse method)}

\begin{fulllineitems}
\phantomsection\label{eqparse:eqparse.baseparse.BaseParse.new_names_restricted}\pysiglinewithargsret{\bfcode{new\_names\_restricted}}{\emph{oNames, mIndex, specifier={[}'`, `'{]}, return\_it=True}}{}
\end{fulllineitems}

\index{new\_order() (eqparse.baseparse.BaseParse method)}

\begin{fulllineitems}
\phantomsection\label{eqparse:eqparse.baseparse.BaseParse.new_order}\pysiglinewithargsret{\bfcode{new\_order}}{\emph{*argv}}{}
\end{fulllineitems}

\index{open\_file() (eqparse.baseparse.BaseParse method)}

\begin{fulllineitems}
\phantomsection\label{eqparse:eqparse.baseparse.BaseParse.open_file}\pysiglinewithargsret{\bfcode{open\_file}}{\emph{file\_n\_str\_key}}{}
\end{fulllineitems}

\index{order\_index() (eqparse.baseparse.BaseParse method)}

\begin{fulllineitems}
\phantomsection\label{eqparse:eqparse.baseparse.BaseParse.order_index}\pysiglinewithargsret{\bfcode{order\_index}}{\emph{index\_to\_order}, \emph{order\_by=None}}{}
\end{fulllineitems}

\index{pattern\_change() (eqparse.baseparse.BaseParse method)}

\begin{fulllineitems}
\phantomsection\label{eqparse:eqparse.baseparse.BaseParse.pattern_change}\pysiglinewithargsret{\bfcode{pattern\_change}}{\emph{gIndex}, \emph{orig}, \emph{*argv}}{}
\end{fulllineitems}

\index{pattern\_write() (eqparse.baseparse.BaseParse method)}

\begin{fulllineitems}
\phantomsection\label{eqparse:eqparse.baseparse.BaseParse.pattern_write}\pysiglinewithargsret{\bfcode{pattern\_write}}{\emph{w\_file}, \emph{w\_data\_tuple}, \emph{w\_str\_tuple}, \emph{w\_lib=None}, \emph{mIndex=None}, \emph{end\_wrr\_='n'}, \emph{beg\_wrr\_='`}}{}
Write to file output each of the required variables with a similar coded pattern
\begin{quote}\begin{description}
\item[{Parameters}] \leavevmode\begin{itemize}
\item {} 
\textbf{w\_file} (\emph{str}) -- file to write to specified by string used in open\_file function

\item {} 
\textbf{w\_data\_tuple} (\emph{list}) -- combination of strings (if same for each variable) and lists (each with same size) that will be printed in consecutive order

\item {} 
\textbf{w\_str\_tuple} (\emph{str}) -- 

\item {} 
\textbf{w\_lib} -- post-fix the returned property with a string.

\item {} 
\textbf{mIndex} (\emph{list}) -- specify index of variables to go through.

\end{itemize}

\end{description}\end{quote}

\end{fulllineitems}

\index{rasterise\_dict\_modifiers() (eqparse.baseparse.BaseParse method)}

\begin{fulllineitems}
\phantomsection\label{eqparse:eqparse.baseparse.BaseParse.rasterise_dict_modifiers}\pysiglinewithargsret{\bfcode{rasterise\_dict\_modifiers}}{\emph{refi}}{}
\end{fulllineitems}

\index{replace\_operations() (eqparse.baseparse.BaseParse method)}

\begin{fulllineitems}
\phantomsection\label{eqparse:eqparse.baseparse.BaseParse.replace_operations}\pysiglinewithargsret{\bfcode{replace\_operations}}{\emph{ops=None}}{}
\end{fulllineitems}

\index{search\_and\_replace() (eqparse.baseparse.BaseParse method)}

\begin{fulllineitems}
\phantomsection\label{eqparse:eqparse.baseparse.BaseParse.search_and_replace}\pysiglinewithargsret{\bfcode{search\_and\_replace}}{\emph{origContainer}, \emph{repNames}, \emph{origNames=None}, \emph{return\_it=True}}{}
Search and replace the variables of a container of equations (normally the 
RHS or initial condition) to specified new names. All containers
should have same length and coincide to variable  name properties
\begin{quote}\begin{description}
\item[{Parameters}] \leavevmode\begin{itemize}
\item {} 
\textbf{origContainer} -- data of formular/equations containing variables one wants to change

\item {} 
\textbf{repNames} -- the names one wants to replace to parallel to :param \emph{origNames}:

\item {} 
\textbf{origNames} -- the original names to replace (default - original variable names)

\end{itemize}

\end{description}\end{quote}

\end{fulllineitems}

\index{vectorise\_name() (eqparse.baseparse.BaseParse method)}

\begin{fulllineitems}
\phantomsection\label{eqparse:eqparse.baseparse.BaseParse.vectorise_name}\pysiglinewithargsret{\bfcode{vectorise\_name}}{\emph{mIndex}, \emph{specifier}, \emph{names\_list=None}, \emph{return\_it=True}}{}
\end{fulllineitems}

\index{write() (eqparse.baseparse.BaseParse method)}

\begin{fulllineitems}
\phantomsection\label{eqparse:eqparse.baseparse.BaseParse.write}\pysiglinewithargsret{\bfcode{write}}{\emph{fileAttr}, \emph{strAttr}, \emph{end\_wrr\_='n'}, \emph{beg\_wrr\_='`}}{}
\end{fulllineitems}

\index{write\_comment() (eqparse.baseparse.BaseParse method)}

\begin{fulllineitems}
\phantomsection\label{eqparse:eqparse.baseparse.BaseParse.write_comment}\pysiglinewithargsret{\bfcode{write\_comment}}{\emph{fileAttr}, \emph{strAttr}, \emph{end\_wrr\_='n'}, \emph{beg\_wrr\_='`}}{}
\end{fulllineitems}


\end{fulllineitems}



\section{\texttt{cppparser} module}
\label{eqparse:module-eqparse.cppparser}\label{eqparse:cppparser-module}\index{eqparse.cppparser (module)}\index{CppParser (class in eqparse.cppparser)}

\begin{fulllineitems}
\phantomsection\label{eqparse:eqparse.cppparser.CppParser}\pysiglinewithargsret{\strong{class }\code{eqparse.cppparser.}\bfcode{CppParser}}{\emph{Lib}}{}
Bases: {\hyperref[eqparse:eqparse.baseparse.BaseParse]{\code{eqparse.baseparse.BaseParse}}}
\index{cpp\_original() (eqparse.cppparser.CppParser method)}

\begin{fulllineitems}
\phantomsection\label{eqparse:eqparse.cppparser.CppParser.cpp_original}\pysiglinewithargsret{\bfcode{cpp\_original}}{}{}
\end{fulllineitems}


\end{fulllineitems}



\section{\texttt{createlibrary} module}
\label{eqparse:createlibrary-module}\label{eqparse:module-eqparse.createlibrary}\index{eqparse.createlibrary (module)}
create Library of variables : Read saved model file (current formats: csv, )
\index{CreateLibrary (class in eqparse.createlibrary)}

\begin{fulllineitems}
\phantomsection\label{eqparse:eqparse.createlibrary.CreateLibrary}\pysiglinewithargsret{\strong{class }\code{eqparse.createlibrary.}\bfcode{CreateLibrary}}{\emph{g\_abspath\_title}, \emph{name\_of\_file\_list}}{}
Bases: \code{object}

Before calling a parser module, one must call this class with the csv
files to be parsed, organised and managed into a dictionary for easy look-up
and readability for the different parsers. Currently this module only accepts 
CSV files. In the future anticipate other possible formats such as XML and
possibly a GUI.

Examples are included that parse example ode problems into several formats

OK
\index{add\_index() (eqparse.createlibrary.CreateLibrary method)}

\begin{fulllineitems}
\phantomsection\label{eqparse:eqparse.createlibrary.CreateLibrary.add_index}\pysiglinewithargsret{\bfcode{add\_index}}{\emph{define\_dict\_keys}, \emph{cond\_on=\{\}}, \emph{cond\_not\_on=\{\}}}{}
\end{fulllineitems}

\index{add\_variable() (eqparse.createlibrary.CreateLibrary method)}

\begin{fulllineitems}
\phantomsection\label{eqparse:eqparse.createlibrary.CreateLibrary.add_variable}\pysiglinewithargsret{\bfcode{add\_variable}}{\emph{mDict}}{}
\end{fulllineitems}

\index{change\_variable() (eqparse.createlibrary.CreateLibrary method)}

\begin{fulllineitems}
\phantomsection\label{eqparse:eqparse.createlibrary.CreateLibrary.change_variable}\pysiglinewithargsret{\bfcode{change\_variable}}{\emph{mname}, \emph{mval}}{}
\end{fulllineitems}

\index{complete() (eqparse.createlibrary.CreateLibrary method)}

\begin{fulllineitems}
\phantomsection\label{eqparse:eqparse.createlibrary.CreateLibrary.complete}\pysiglinewithargsret{\bfcode{complete}}{}{}
\end{fulllineitems}

\index{copy\_lib() (eqparse.createlibrary.CreateLibrary method)}

\begin{fulllineitems}
\phantomsection\label{eqparse:eqparse.createlibrary.CreateLibrary.copy_lib}\pysiglinewithargsret{\bfcode{copy\_lib}}{}{}
\end{fulllineitems}

\index{copy\_lib\_vec() (eqparse.createlibrary.CreateLibrary method)}

\begin{fulllineitems}
\phantomsection\label{eqparse:eqparse.createlibrary.CreateLibrary.copy_lib_vec}\pysiglinewithargsret{\bfcode{copy\_lib\_vec}}{}{}
\end{fulllineitems}

\index{create\_data() (eqparse.createlibrary.CreateLibrary method)}

\begin{fulllineitems}
\phantomsection\label{eqparse:eqparse.createlibrary.CreateLibrary.create_data}\pysiglinewithargsret{\bfcode{create\_data}}{}{}
\end{fulllineitems}

\index{dup2() (eqparse.createlibrary.CreateLibrary method)}

\begin{fulllineitems}
\phantomsection\label{eqparse:eqparse.createlibrary.CreateLibrary.dup2}\pysiglinewithargsret{\bfcode{dup2}}{\emph{n}}{}
\end{fulllineitems}

\index{function\_order() (eqparse.createlibrary.CreateLibrary method)}

\begin{fulllineitems}
\phantomsection\label{eqparse:eqparse.createlibrary.CreateLibrary.function_order}\pysiglinewithargsret{\bfcode{function\_order}}{\emph{*argv}}{}
\end{fulllineitems}

\index{get\_index\_dependency() (eqparse.createlibrary.CreateLibrary method)}

\begin{fulllineitems}
\phantomsection\label{eqparse:eqparse.createlibrary.CreateLibrary.get_index_dependency}\pysiglinewithargsret{\bfcode{get\_index\_dependency}}{\emph{m\_key\_to\_order='init-value'}, \emph{resp\_order\_val='name'}, \emph{cond\_remove=\{\}}, \emph{cond\_include=\{\}}}{}
\end{fulllineitems}

\index{set\_directory() (eqparse.createlibrary.CreateLibrary method)}

\begin{fulllineitems}
\phantomsection\label{eqparse:eqparse.createlibrary.CreateLibrary.set_directory}\pysiglinewithargsret{\bfcode{set\_directory}}{}{}
Set directory where file specific parser files are saved to, 
the directory can be a relative path and not necessarily an absolute path
\begin{quote}\begin{description}
\item[{Parameters}] \leavevmode
\textbf{dir\_str} -- string of path to directory

\end{description}\end{quote}

\end{fulllineitems}

\index{table (eqparse.createlibrary.CreateLibrary attribute)}

\begin{fulllineitems}
\phantomsection\label{eqparse:eqparse.createlibrary.CreateLibrary.table}\pysigline{\bfcode{table}\strong{ = {[}{]}}}
\end{fulllineitems}


\end{fulllineitems}

\index{change\_a\_type() (in module eqparse.createlibrary)}

\begin{fulllineitems}
\phantomsection\label{eqparse:eqparse.createlibrary.change_a_type}\pysiglinewithargsret{\code{eqparse.createlibrary.}\bfcode{change\_a\_type}}{\emph{Lib}, \emph{vName}, \emph{vKey}, \emph{vToo}}{}
\end{fulllineitems}



\section{\texttt{matlabparser} module}
\label{eqparse:matlabparser-module}\label{eqparse:module-eqparse.matlabparser}\index{eqparse.matlabparser (module)}\index{MatlabParser (class in eqparse.matlabparser)}

\begin{fulllineitems}
\phantomsection\label{eqparse:eqparse.matlabparser.MatlabParser}\pysiglinewithargsret{\strong{class }\code{eqparse.matlabparser.}\bfcode{MatlabParser}}{\emph{Lib}}{}
Bases: {\hyperref[eqparse:eqparse.baseparse.BaseParse]{\code{eqparse.baseparse.BaseParse}}}

To create matlab parsed file
\index{FSA\_cvodes() (eqparse.matlabparser.MatlabParser method)}

\begin{fulllineitems}
\phantomsection\label{eqparse:eqparse.matlabparser.MatlabParser.FSA_cvodes}\pysiglinewithargsret{\bfcode{FSA\_cvodes}}{}{}
\end{fulllineitems}

\index{FSA\_ode() (eqparse.matlabparser.MatlabParser method)}

\begin{fulllineitems}
\phantomsection\label{eqparse:eqparse.matlabparser.MatlabParser.FSA_ode}\pysiglinewithargsret{\bfcode{FSA\_ode}}{}{}
Create FSA matlab ODE file (ode.m) to be used with \code{multi\_runfile} and \code{multi\_runfile\_slider} modules

\end{fulllineitems}

\index{inset\_runfile\_slider() (eqparse.matlabparser.MatlabParser method)}

\begin{fulllineitems}
\phantomsection\label{eqparse:eqparse.matlabparser.MatlabParser.inset_runfile_slider}\pysiglinewithargsret{\bfcode{inset\_runfile\_slider}}{\emph{newSubDirectory='`}, \emph{file\_ext='`}, \emph{no\_guess=None}, \emph{yes\_guess=None}}{}
\end{fulllineitems}

\index{multi\_odefile() (eqparse.matlabparser.MatlabParser method)}

\begin{fulllineitems}
\phantomsection\label{eqparse:eqparse.matlabparser.MatlabParser.multi_odefile}\pysiglinewithargsret{\bfcode{multi\_odefile}}{\emph{newSubDirectory='`}, \emph{file\_ext='`}, \emph{no\_guess=None}, \emph{yes\_guess=None}, \emph{wr\_modifiers='`}}{}
Create matlab ODE file (ode.m) to be used with \code{multi\_runfile} and \code{multi\_runfile\_slider} modules

\end{fulllineitems}

\index{multi\_paramfile() (eqparse.matlabparser.MatlabParser method)}

\begin{fulllineitems}
\phantomsection\label{eqparse:eqparse.matlabparser.MatlabParser.multi_paramfile}\pysiglinewithargsret{\bfcode{multi\_paramfile}}{\emph{newSubDirectory='`}, \emph{file\_ext='`}, \emph{no\_guess=None}, \emph{yes\_guess=None}}{}
\end{fulllineitems}

\index{multi\_runfile() (eqparse.matlabparser.MatlabParser method)}

\begin{fulllineitems}
\phantomsection\label{eqparse:eqparse.matlabparser.MatlabParser.multi_runfile}\pysiglinewithargsret{\bfcode{multi\_runfile}}{\emph{newSubDirectory='`}, \emph{file\_ext='`}, \emph{no\_guess=None}, \emph{yes\_guess=None}, \emph{wr\_modifiers='`}}{}
Create run file (run.m) that uses the ode file created from running alongside with \code{multi\_odefile} module. This format is currently set for MATLAB defined implicit ode function - ode15s.
\begin{quote}\begin{description}
\item[{Parameters}] \leavevmode\begin{itemize}
\item {} 
\textbf{newSubDirectory} -- 

\item {} 
\textbf{file\_ext} -- 

\item {} 
\textbf{no\_guess} -- 

\item {} 
\textbf{yes\_guess} -- 

\item {} 
\textbf{wr\_modifiers} -- 

\end{itemize}

\end{description}\end{quote}

\end{fulllineitems}

\index{multi\_runfile\_slider() (eqparse.matlabparser.MatlabParser method)}

\begin{fulllineitems}
\phantomsection\label{eqparse:eqparse.matlabparser.MatlabParser.multi_runfile_slider}\pysiglinewithargsret{\bfcode{multi\_runfile\_slider}}{\emph{newSubDirectory='`}, \emph{file\_ext='`}, \emph{no\_guess=None}, \emph{yes\_guess=None}}{}
\end{fulllineitems}

\index{p\_est\_main() (eqparse.matlabparser.MatlabParser method)}

\begin{fulllineitems}
\phantomsection\label{eqparse:eqparse.matlabparser.MatlabParser.p_est_main}\pysiglinewithargsret{\bfcode{p\_est\_main}}{\emph{param\_to\_est}}{}
\end{fulllineitems}

\index{solve\_single() (eqparse.matlabparser.MatlabParser method)}

\begin{fulllineitems}
\phantomsection\label{eqparse:eqparse.matlabparser.MatlabParser.solve_single}\pysiglinewithargsret{\bfcode{solve\_single}}{}{}
generate single MATLAB (single\_generic.m) file to solve equations. Runs the Euler finite differece scheme to model with  dt=5 (ms)

\end{fulllineitems}

\index{supp\_initialise\_data() (eqparse.matlabparser.MatlabParser method)}

\begin{fulllineitems}
\phantomsection\label{eqparse:eqparse.matlabparser.MatlabParser.supp_initialise_data}\pysiglinewithargsret{\bfcode{supp\_initialise\_data}}{}{}
\end{fulllineitems}

\index{supp\_input\_ic() (eqparse.matlabparser.MatlabParser method)}

\begin{fulllineitems}
\phantomsection\label{eqparse:eqparse.matlabparser.MatlabParser.supp_input_ic}\pysiglinewithargsret{\bfcode{supp\_input\_ic}}{}{}
\end{fulllineitems}

\index{vectorised\_to\_readable() (eqparse.matlabparser.MatlabParser method)}

\begin{fulllineitems}
\phantomsection\label{eqparse:eqparse.matlabparser.MatlabParser.vectorised_to_readable}\pysiglinewithargsret{\bfcode{vectorised\_to\_readable}}{}{}
\end{fulllineitems}


\end{fulllineitems}



\section{\texttt{smc\_helper\_functions} module}
\label{eqparse:module-eqparse.smc_helper_functions}\label{eqparse:smc-helper-functions-module}\index{eqparse.smc\_helper\_functions (module)}\index{error() (in module eqparse.smc\_helper\_functions)}

\begin{fulllineitems}
\phantomsection\label{eqparse:eqparse.smc_helper_functions.error}\pysiglinewithargsret{\code{eqparse.smc\_helper\_functions.}\bfcode{error}}{\emph{err\_str}}{}
\end{fulllineitems}



\section{\texttt{timer} module}
\label{eqparse:timer-module}\label{eqparse:module-eqparse.timer}\index{eqparse.timer (module)}\index{Timer (class in eqparse.timer)}

\begin{fulllineitems}
\phantomsection\label{eqparse:eqparse.timer.Timer}\pysigline{\strong{class }\code{eqparse.timer.}\bfcode{Timer}}~\index{finish() (eqparse.timer.Timer method)}

\begin{fulllineitems}
\phantomsection\label{eqparse:eqparse.timer.Timer.finish}\pysiglinewithargsret{\bfcode{finish}}{\emph{m\_str=None}}{}
\end{fulllineitems}

\index{start() (eqparse.timer.Timer method)}

\begin{fulllineitems}
\phantomsection\label{eqparse:eqparse.timer.Timer.start}\pysiglinewithargsret{\bfcode{start}}{}{}
\end{fulllineitems}


\end{fulllineitems}



\section{\texttt{xppautparser} module}
\label{eqparse:xppautparser-module}\label{eqparse:module-eqparse.xppautparser}\index{eqparse.xppautparser (module)}\index{XppautParser (class in eqparse.xppautparser)}

\begin{fulllineitems}
\phantomsection\label{eqparse:eqparse.xppautparser.XppautParser}\pysiglinewithargsret{\strong{class }\code{eqparse.xppautparser.}\bfcode{XppautParser}}{\emph{Lib}}{}
Bases: {\hyperref[eqparse:eqparse.baseparse.BaseParse]{\code{eqparse.baseparse.BaseParse}}}
\index{parse\_one() (eqparse.xppautparser.XppautParser method)}

\begin{fulllineitems}
\phantomsection\label{eqparse:eqparse.xppautparser.XppautParser.parse_one}\pysiglinewithargsret{\bfcode{parse\_one}}{\emph{concateq\_unordered=}\optional{}}{}
\end{fulllineitems}

\index{printarray() (eqparse.xppautparser.XppautParser method)}

\begin{fulllineitems}
\phantomsection\label{eqparse:eqparse.xppautparser.XppautParser.printarray}\pysiglinewithargsret{\bfcode{printarray}}{\emph{c}, \emph{specific=None}}{}
\end{fulllineitems}

\index{set\_temp\_ic\_name() (eqparse.xppautparser.XppautParser method)}

\begin{fulllineitems}
\phantomsection\label{eqparse:eqparse.xppautparser.XppautParser.set_temp_ic_name}\pysiglinewithargsret{\bfcode{set\_temp\_ic\_name}}{}{}
\end{fulllineitems}

\index{truncate\_ics() (eqparse.xppautparser.XppautParser method)}

\begin{fulllineitems}
\phantomsection\label{eqparse:eqparse.xppautparser.XppautParser.truncate_ics}\pysiglinewithargsret{\bfcode{truncate\_ics}}{}{}
\end{fulllineitems}

\index{xpp\_search\_and\_define() (eqparse.xppautparser.XppautParser method)}

\begin{fulllineitems}
\phantomsection\label{eqparse:eqparse.xppautparser.XppautParser.xpp_search_and_define}\pysiglinewithargsret{\bfcode{xpp\_search\_and\_define}}{}{}
\end{fulllineitems}


\end{fulllineitems}



\section{Module contents}
\label{eqparse:module-eqparse}\label{eqparse:module-contents}\index{eqparse (module)}
eqparse import module initialise


\renewcommand{\indexname}{Python Module Index}
\begin{theindex}
\def\bigletter#1{{\Large\sffamily#1}\nopagebreak\vspace{1mm}}
\bigletter{e}
\item {\texttt{eqparse}}, \pageref{eqparse:module-eqparse}
\item {\texttt{eqparse.\_\_init\_\_}}, \pageref{eqparse:module-eqparse.__init__}
\item {\texttt{eqparse.baseparse}}, \pageref{eqparse:module-eqparse.baseparse}
\item {\texttt{eqparse.cppparser}}, \pageref{eqparse:module-eqparse.cppparser}
\item {\texttt{eqparse.createlibrary}}, \pageref{eqparse:module-eqparse.createlibrary}
\item {\texttt{eqparse.matlabparser}}, \pageref{eqparse:module-eqparse.matlabparser}
\item {\texttt{eqparse.smc\_helper\_functions}}, \pageref{eqparse:module-eqparse.smc_helper_functions}
\item {\texttt{eqparse.timer}}, \pageref{eqparse:module-eqparse.timer}
\item {\texttt{eqparse.xppautparser}}, \pageref{eqparse:module-eqparse.xppautparser}
\end{theindex}

\renewcommand{\indexname}{Index}
\printindex
\end{document}
